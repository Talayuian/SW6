\Section{Conclusion}
The current prototype does not fulfil the softer requirements, because of limited time to acquire the needed material and solving hardware issues.
However all the harder requirements defined in Section \ref{sec:Requirement Specification} have been fulfilled by the software, as can be seen by the tests performed in Chapter \ref{ch:Test}.
Therefore this system has been successful, despite leaving room for further development.
These developments include the \textit{Should Have} in Section \ref{sec:Requirement Specification}, along with the \textit{Could Have}.
This can therefore be seen as a successful way of constructing a system, capable of sorting \textbf{LEGO} bricks based on both colour and size, while also working on an embedded system, thus fulfilling Section \ref{sec:Problem Definition}.

%problemdefinition
%    \item How could a sorting machine be constructed, capable of sorting LEGO bricks based on both colour and size?
%    \item How can such a system be designed and built on an embedded system?

%    \item \textbf{Sorting of Size}\\
%        Selecting the proper object based on a size criteria given by the blueprint, and rejecting it if any deformity or size not specified by the blueprint is found.
%        %If the object's measurement is required but not the next on the output list, it will be returned to the back of the conveyor line.
        
%    \item \textbf{Sorting of Colour}\\
%        Selecting the proper object based on colour criteria given by the blueprint, and rejecting it if the colour is not the specified in a blueprint, or returning it to the back of the line. 
        
%    \item \textbf{Conveyor Belt}\\
%        A functional conveyor belt capable of transporting at least ten objects---through the sensor area---per minute, without any objects falling off, turning over, or providing inaccurate information for the sensors.
    
%    \item \textbf{Sort by a Sorting List}\\
%        The product should be capable of sorting bricks by the order in which they appear in a given input list.
%    \end{itemize}

% AND