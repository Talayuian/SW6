\Section{Reflection}
Here the work process will be evaluated.
It will attempt to explain the difficulties there was in the work progress, how it affected the work-flow, and a better way or possible solution.

\Subsection{SCRUM}
While there was an attempt at utilising SCRUM, some parts of it where not properly utilised.
This primarily affected the SCRUM meetings.
The reason for this was that many times the meeting repeated themselves, making each meeting feel redundant.
After slowly phasing these meetings out however, there was a partial loss of direction in the work flow.
The better method of doing this, would be to instead evaluate the meetings and potentially hold them only every other day.

Another part that could have been better utilised, would be the \textit{Burn-Down Chart}.
The chart was not used in the start of the project, but was instead introduced after approximately one month.
And while it did prove useful with locating potential problems in the time planning, it was not properly updated.
This indicates a problem with the work culture, meaning that the best fix for this problem, would be to force the individuals to update it consistently.

Both the SCRUM board and the work sprints however proved to be a useful tool for delegating work.
These helped compensate for the lack of direction, which were caused by not having daily SCRUM meetings.

Overall the SCRUM methods have in general been useful for the project, despite not being properly utilised.
This is therefore something which the individuals of the group will most likely try again, so as to gain the proper skills in using it.

\Subsection{Pair-Working}
For this section pair-working relates to 2 processes: \textit{Pair-Programming} and \textit{Pair-Writing}.
Both where utilised throughout the entire project, with the exception of the last part due to time-constraints.
One issue that occurred with both methods, was that in the beginning pair-cliques took shape.
This meant that there was problems ensuring all individuals knew everything about the project.
The way this problem will be fixed in later projects, will be by including a specific rotation of who will work with whom.
This problem has since subsided as the solution was attempted relatively early; however, it could be improved by assigning a \textit{task leader}---someone who is primarily responsible for a given task---for whom the partner is changed at every rotation cycle. 
That is, each task must have a leader who not only ensures the task gets finished, but also shares the knowledge acquired in an easy and concise manner.
The leader of each task will always stay the same, no matter how much progress has been made on the task.
Their partner however will change during the development, to improve the amount of knowledge sharing.
This leader system solves the problem of everyone working on a task they were not responsible for.
Having a single responsible party ensures there is knowledge sharing on the different tasks.

Pair writing had the same benefits as pair programming, in that minor problems where easily found.
It also made larger problems more identifiable, as the second person often saw them during the writing period.
One thing that helped tremendously with pair writing was the fact the report was written online.
This meant that both parties would be capable of writing, as opposed to delegating one individual to only read.

\Subsection{Group Contract}
Before any work was begun, all members of the group signed a group contract detailing what is expected of each member.
This contract consisted of a designated meeting time, how the daily meeting---the one which was phased out---should go, communication, etc.

The role delegation which the contract gave did give a slight benefit, as it enforced the way which people wanted to work.
However some roles---such as dedicated note taker during meetings---where not properly enforced. 
This did not have any impact on the amount of work done, but it was a slight inconvenience to the ones who repeatedly ended up taking notes when the task should have been shared.
%TAnD