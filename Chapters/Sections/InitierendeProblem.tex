\Chapter{Introduction} 
Automation is an important part of industry in the current market, as it is often cheaper to use machinery than human labour, while also potentially reducing the amount of errors.
Each automated subsystem relies on acquiring the proper materials; to ensure this, a proper sorting system is preferred, these menial task can either be entirely or partly automated.

This project attempts to explore the basics of automated sorting, by using simple standardised materials.
For this project the material used, is \textbf{LEGO} bricks.
The purpose of this system will be to sort materials by various characteristics, such as size, colour, or similar; only accepting expected \textbf{LEGO} bricks.

To ensure cost-efficiency, this is done with an embedded system, due to their low price and electrical usage.

\Section{The Initial Problem} 
How can a system capable of sorting \textbf{LEGO} bricks be constructed in a cost efficient way?

%TAND
